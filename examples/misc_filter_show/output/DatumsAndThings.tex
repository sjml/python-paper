\documentclass[
    12pt,
    letterpaper,
    oneside,
    noraggedright
]{turabian-researchpaper}

% condition usage
\usepackage{ifthen}

% Tempora is a clone of Times Roman that
%   includes Greek and Cyrillic; should do
%   for most STM needs, but may need a Hebrew
%   solution some day.
\usepackage{fontspec}
\defaultfontfeatures{Scale=MatchLowercase}
\defaultfontfeatures[\rmfamily]{Ligatures=TeX,Scale=1}

\setmainfont{Tempora}
\setmonofont{Consolas}

% font and typesetting niceties
\usepackage{upquote}
\usepackage{microtype}
\setlength{\emergencystretch}{3em}


\title{{\textbf{Iesus Nazarenus}, Rex Iudaeorum}}
\subtitle{{\emph{τι εστιν αληθεια?}}}
\author{{Πόντιος Πιλᾶτος}}

\PassOptionsToPackage{hyphens}{url}
\usepackage[hyperfootnotes=false,hidelinks,unicode]{hyperref}
\hypersetup{
    pdftitle={{\textbf{Iesus Nazarenus}, Rex Iudaeorum}: {\emph{τι εστιν αληθεια?}} },
    pdfauthor={{Πόντιος Πιλᾶτος}},
    pdflang={en-US},
    pdfcreator={SJML Paper v0.3.0},
    % pdfproducer={},
}
% for nicer linebreak placements in URLs
\usepackage{xurl}
% for underlines on links
\usepackage{href-ul}

% page numbers centered in footer
\pagestyle{plain}

% tables
\usepackage{longtable}
\usepackage{booktabs}
\usepackage{array}

% default tables are opulently tall
\let\oldlongtable\longtable
\def\longtable{\onehalfspacing \oldlongtable}

% images
\usepackage{graphicx}
\makeatletter
\def\maxwidth{\ifdim\Gin@nat@width>\linewidth\linewidth\else\Gin@nat@width\fi}
\def\maxheight{\ifdim\Gin@nat@height>\textheight\textheight\else\Gin@nat@height\fi}
\makeatother

\graphicspath{{../content}} % image paths relative to content directory
% image aspect ratios stay correct, able to use [caption](./image.png){width=2in}
\setkeys{Gin}{width=\maxwidth,height=\maxheight,keepaspectratio}

% captions
\makeatletter
%% figure captions (centered and italicized)
\setlength\abovecaptionskip{\z@}
\setlength\belowcaptionskip{-.5\baselineskip}
\long\def\@makecaption#1#2{%
	\vskip\abovecaptionskip
	\center\small\emph{#1. #2}\par
	\vskip\belowcaptionskip
}
%% table captions (why can't these be the same?!)
\patchcmd{\LT@makecaption}%
   {#1{#2: }#3}
   {#1\small\emph{#2. #3}}{}{}
\makeatother

% lists should be single-spaced
\let\oldenumerate\enumerate
\def\enumerate{\oldenumerate \singlespacing}
\let\olditemize\itemize
\def\itemize{\olditemize \singlespacing}

% tight list command required by pandoc output
\makeatletter
\providecommand{\tightlist}{%
    \ifthenelse{\equal{\the\@listdepth}{1}}
    {}
    {\vspace{-\baselineskip}}
}
\makeatother

% tighter footnote spacing
\usepackage{footnote}
\makesavenoteenv{figure}
\makesavenoteenv{longtable}
\DeclareRobustCommand{\href}[2]{#2\footnote{\url{#1}}}
\setlength{\footnotesep}{16pt}

% section headers are left-aligned and slightly larger font
\let\oldsection=\section
\renewcommand\section[1]{\oldsection{\texorpdfstring{\large\raggedright{#1}}{#1}}}
\let\oldsubsection=\subsection
\renewcommand\subsection[1]{\oldsubsection{\texorpdfstring{\large\raggedright\emph{#1}}{#1}}}
\let\oldsubsubsection=\subsubsection
\renewcommand\subsubsection[1]{\oldsubsubsection{\texorpdfstring{\raggedright\emph{#1}}{#1}}}

% standard bibliography environment taken from default pandoc template
%   except where noted
\newlength{\cslhangindent}
\setlength{\cslhangindent}{1.5em}
\newlength{\csllabelwidth}
\setlength{\csllabelwidth}{3em}
\newlength{\cslentryspacingunit}
\setlength{\cslentryspacingunit}{\parskip}
\newenvironment{CSLReferences}[2]
 {
\newpage % start on new page
\centerline{\underline{Bibliography}} % label it
 \setlength{\parindent}{0pt}
 \singlespacing
  \let\oldpar\par
  \def\par{\hangindent=\cslhangindent\oldpar}
  \setlength{\parskip}{1em}
 }
{
}
\usepackage{calc}
\newcommand{\CSLBlock}[1]{#1\hfill\break}
\newcommand{\CSLLeftMargin}[1]{\parbox[t]{\csllabelwidth}{#1}}
\newcommand{\CSLRightInline}[1]{\parbox[t]{\linewidth - \csllabelwidth}{#1}\break}
\newcommand{\CSLIndent}[1]{\hspace{\cslhangindent}#1}


% where the magic happens
\begin{document}
    \begin{center}

    \thispagestyle{empty}
    \vspace*{1in}
    \begin{singlespace}
        {\textbf{Iesus Nazarenus}, Rex Iudaeorum}:\\
    {\emph{τι εστιν αληθεια?}}
        \end{singlespace}
    \vspace{2in - \baselineskip}

    by
    \vspace{2in - \baselineskip}

    {Πόντιος Πιλᾶτος}

    \vspace{2in - \baselineskip}

    \singlespace{
        {Imperator Caesar \emph{Divi filius} Augustus} \\
        {SPQR CI} --- {Lex Iniusta} \\
        {April 3, A.U.C. 786}
    }
    \end{center}
    \newpage
    \setcounter{page}{1}

Citing the Vulgate the first time produces a footnote, as in ``In
principio erat Verbum, et Verbum erat apud Deum, et Deus erat Verbum.''
{(John. 1:1)\footnote{Bonifatius Fischer and Robert Weber, eds.,
  \emph{Biblia Sacra: Iuxta Vulgatam Versionem}, Ed. quartam emendatam
  (Stuttgart: Deutsche Bibelgesellschaft, 1994).}} But a second citation
becomes parenthetical with an italicized attribution as in ``Qui non
diligit, non novit Deum: quoniam Deus caritas est.'' {(1 John. 4:8{
\emph{Vulgatam}})}

The first time the Catechism gets cited, it's the full deal.\footnote{\emph{Catechism
  of the Catholic Church}, 2nd ed. (Vatican City; Washington, D.C.:
  Libreria Editrice Vaticana, 2019), p45.} But what does Tommy
Quine-Quine have to say about the same subject?\footnote{Thomas Aquinas,
  \emph{Summa Theologica}, trans. Fathers of the English Dominican
  Province, Second and Revised Edition (New Advent, 1920), p48,
  \url{https://www.newadvent.org/summa/}.} CCC with a stirring
rebuttal!\footnote{\emph{CCC}, 49.} (Notice that it just becomes ``CCC''
on subsequents.)

That one was easy, because the
\href{https://citationstyles.org/}{Citation Style Language} (note URLs
for links go in footnotes, but only in LaTeX) has ready support for
short titles, and since the Catechism doesn't have an \emph{author} to
list, it works out. What's trickier is the USCCB, which wants to get
referenced by its full name the first time,\footnote{United States
  Conference of Catholic Bishops, \emph{Program of Priestly Formation},
  5th ed. (Washington, D.C: United States Conference of Catholic
  Bishops, 2006).} but then afterwards gets ``institutionally
abbreviated.''\footnote{USCCB, \emph{Program of Priestly Formation}.}
(Support for author short names is supposed to come in CSL 1.1, but
there's no timetable on that, so gotta use
\href{https://github.com/sjml/paper/tree/main/paper/resources/project_template/.paper_resources/filters}{my
little Lua filters} to get it done.) It works with multiple citations,
too.\footnote{\emph{CCC}, p45; USCCB, \emph{Program of Priestly
  Formation}, §17.}

For Papal Encyclicals, we need to make sure the capitalization of the
Latin name stays consistent, so this cite\footnote{Pope Francis,
  {Encyclical on Fraternity and Social Friendship \emph{Fratelli
  tutti},} October 3, 2020, §42, at The Holy See,
  \url{https://www.vatican.va/content/francesco/en/encyclicals/documents/papa-francesco_20201003_enciclica-fratelli-tutti.html}.}
should mention ``\emph{Fratelli tutti}'' but not in title case. Should
also happen for its second mention\footnote{{\emph{Fratelli tutti},}
  §56.} where it only goes by the Latin name. No quotes around any of it
in the citation.

In the end, we give Tommy the final word,\footnote{\emph{ST} I-II, Q.
  12.} and note the author drop from his subsequent citation.

\hypertarget{refs}{}
\begin{CSLReferences}{1}{0}
\leavevmode\vadjust pre{\hypertarget{ref-aquinasSumma}{}}%
Aquinas, Thomas. \emph{Summa Theologica}. Translated by Fathers of the
English Dominican Province. Second and Revised Edition. New Advent,
1920. \url{https://www.newadvent.org/summa/}.

\leavevmode\vadjust pre{\hypertarget{ref-CCC.2017}{}}%
\emph{Catechism of the Catholic Church}. 2nd ed. Vatican City;
Washington, D.C.: Libreria Editrice Vaticana, 2019.

\leavevmode\vadjust pre{\hypertarget{ref-fischerBibliaSacraIuxta1994}{}}%
Fischer, Bonifatius, and Robert Weber, eds. \emph{Biblia Sacra: Iuxta
Vulgatam Versionem}. Ed. quartam emendatam. Stuttgart: Deutsche
Bibelgesellschaft, 1994.

\leavevmode\vadjust pre{\hypertarget{ref-francisFratelliTutti2020}{}}%
Pope Francis. {Encyclical on Fraternity and Social Friendship
\emph{Fratelli tutti},} October 3, 2020. at The Holy See.
\url{https://www.vatican.va/content/francesco/en/encyclicals/documents/papa-francesco_20201003_enciclica-fratelli-tutti.html}.

\leavevmode\vadjust pre{\hypertarget{ref-usccbPPF52006}{}}%
United States Conference of Catholic Bishops. \emph{Program of Priestly
Formation}. 5th ed. Washington, D.C: United States Conference of
Catholic Bishops, 2006.

\end{CSLReferences}

\end{document}
