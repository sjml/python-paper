\documentclass[
    12pt,
    letterpaper,
    oneside,
    noraggedright
]{turabian-researchpaper}

% Times Roman fonts
\usepackage{newtxtext}

\title{{Iesus Nazarenus, Rex Iudaeorum}}
\author{{Ποντιος Πιλατος}}

\usepackage[hyperfootnotes=false,hidelinks,unicode]{hyperref}
\hypersetup{
    pdftitle={{Iesus Nazarenus, Rex Iudaeorum}},
    pdfauthor={{Ποντιος Πιλατος}},
    pdflang={en-US},
    pdfcreator={},
    pdfproducer={SJML Paper v0.0.1},
}
% for nicer linebreak placements in URLs
\usepackage{xurl}

% page numbers centered in footer
\pagestyle{plain}

% images
\usepackage{graphicx}
\graphicspath{{../content/}} % image paths relative to content directory
% image aspect ratios stay correct, able to use [caption](./image.png){width=2in}
\setkeys{Gin}{width=\maxwidth,height=\maxheight,keepaspectratio}

% tables
\usepackage{longtable}
\usepackage{booktabs}

% default tables are opulently tall
\let\oldlongtable\longtable
\def\longtable{\onehalfspacing\vspace{\baselineskip} \oldlongtable}

% captions should be centered and italicized
%% TODO: figure out how to work **with** the turabian-formatting
%%       package's caption stuf instead of fighting it.
\usepackage{caption}
\captionsetup{justification=centering,aboveskip={.5\baselineskip},belowskip={-0.5\baselineskip},font={it}}

% lists should be single-spaced
\let\oldenumerate\enumerate
\def\enumerate{\oldenumerate \singlespacing}
\let\olditemize\itemize
\def\itemize{\olditemize \singlespacing}

% tight lists
\makeatletter
\providecommand{\tightlist}{%
    \ifthenelse{\equal{\the\@listdepth}{1}}
    {}
    {\vspace{-\baselineskip}}
}
\makeatother

% tighter footnote spacing
\setlength{\footnotesep}{16.65pt}

% section headers are left-aligned and slightly larger font
\let\oldsection=\section
\renewcommand\section[1]{\oldsection{\large\raggedright{#1}}}
\let\oldsubsection=\subsection
\renewcommand\subsection[1]{\oldsubsection{\large\raggedright\emph{#1}}}
\let\oldsubsubsection=\subsubsection
\renewcommand\subsubsection[1]{\oldsubsubsection{\raggedright\emph{#1}}}

% standard bibliography environment taken from default pando template
%   except where noted
\newlength{\cslhangindent}
\setlength{\cslhangindent}{1.5em}
\newlength{\csllabelwidth}
\setlength{\csllabelwidth}{3em}
\newlength{\cslentryspacingunit}
\setlength{\cslentryspacingunit}{\parskip}
\newenvironment{CSLReferences}[2]
 {
\newpage % start on new page
\centerline{\underline{Works Cited}} % label it
 \setlength{\parindent}{0pt}
 \singlespacing
  \let\oldpar\par
  \def\par{\hangindent=\cslhangindent\oldpar}
  \setlength{\parskip}{1em}
 }
{
}
\usepackage{calc}
\newcommand{\CSLBlock}[1]{#1\hfill\break}
\newcommand{\CSLLeftMargin}[1]{\parbox[t]{\csllabelwidth}{#1}}
\newcommand{\CSLRightInline}[1]{\parbox[t]{\linewidth - \csllabelwidth}{#1}\break}
\newcommand{\CSLIndent}[1]{\hspace{\cslhangindent}#1}


% where the magic happens
\begin{document}
    \begin{center}

    \thispagestyle{empty}
    \vspace*{1in}
    \begin{singlespace}
        {Iesus Nazarenus, Rex Iudaeorum}
        \end{singlespace}
    \vspace{2in - \baselineskip}

    by
    \vspace{2in - \baselineskip}

    {Ποντιος Πιλατος}

    \vspace{2in - \baselineskip}

    \singlespace{
        {Imperator Caesar Divi filius Augustus} \\
        {SPQR CI} --- {Lex Iniusta} \\
        {April 3, 0033}
    }
    \end{center}
    \newpage
    \setcounter{page}{1}

Citing the Vulgate the first time produces a footnote, as in ``In
principio erat Verbum, et Verbum erat apud Deum, et Deus erat Verbum.''
{(John. 1:1)\footnote{Bonifatius Fischer and Robert Weber, eds.,
  \emph{Biblia Sacra: Iuxta Vulgatam Versionem}, Ed. quartam emendatam
  (Stuttgart: Deutsche Bibelgesellschaft, 1994).}}

But a second citation becomes parenthetical with an italicized
attribution as in ``Qui non diligit, non novit Deum: quoniam Deus
caritas est.'' {(1 John. 4:8{ \emph{Vulgatam}})}

The first time the Catechism gets cited, it's the full deal.\footnote{\emph{Catechism
  of the Catholic Church}, 2nd ed. (Vatican City; Washington, D.C.:
  Libreria Editrice Vaticana, 2019), p45.}

But what does Tommy Quine-Quine have to say about the same
subject?\footnote{Thomas Aquinas, \emph{Summa Theologica}, trans.
  Fathers of the English Dominican Province, Second and Revised Edition
  (New Advent, 1920), p48, \url{https://www.newadvent.org/summa/}.}

CCC with a stirring rebuttal!\footnote{\emph{CCC}, 49.} (Notice that it
just becomes ``CCC'' on subsequents.)

That one was easy, because the
\href{https://citationstyles.org/}{Citation Style Language} has ready
support for short titles, and since the Catechism doesn't have an
\emph{author} to list, it works out. What's trickier is the USCCB, which
wants to get referenced by its full name the first time,\footnote{United
  States Conference of Catholic Bishops, \emph{Program of Priestly
  Formation}, 5th ed. (Washington, D.C: United States Conference of
  Catholic Bishops, 2006).} but then afterwards gets ``institutionally
abbreviated.''\footnote{USCCB, Ibid.} (Support for author short names is
supposed to come in CSL 1.1, but there's no timetable on that, so gotta
use
\href{https://github.com/sjml/paper/tree/main/paper/resources/project_template/.paper_resources/filters}{my
little Lua filters} to get it done.)

Finally, give Tommy the final word,\footnote{\emph{ST} I-II, Q. 12.} and
note the author drop from his subsequent citation.

\hypertarget{refs}{}
\begin{CSLReferences}{1}{0}
\leavevmode\vadjust pre{\hypertarget{ref-aquinasSumma}{}}%
Aquinas, Thomas. \emph{Summa Theologica}. Translated by Fathers of the
English Dominican Province. Second and Revised Edition. New Advent,
1920. \url{https://www.newadvent.org/summa/}.

\leavevmode\vadjust pre{\hypertarget{ref-CCC.2017}{}}%
\emph{Catechism of the Catholic Church}. 2nd ed. Vatican City;
Washington, D.C.: Libreria Editrice Vaticana, 2019.

\leavevmode\vadjust pre{\hypertarget{ref-fischerBibliaSacraIuxta1994}{}}%
Fischer, Bonifatius, and Robert Weber, eds. \emph{Biblia Sacra: Iuxta
Vulgatam Versionem}. Ed. quartam emendatam. Stuttgart: Deutsche
Bibelgesellschaft, 1994.

\leavevmode\vadjust pre{\hypertarget{ref-usccb.PPF52006}{}}%
United States Conference of Catholic Bishops. \emph{Program of Priestly
Formation}. 5th ed. Washington, D.C: United States Conference of
Catholic Bishops, 2006.

\end{CSLReferences}

\end{document}
